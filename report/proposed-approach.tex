\chapter{Background}
This chapter will outline the approaches considered to accomplish this project. The goal of this project was to develop a modular platform for sensing the wireless spectrum through the use of a software defined radio (SDR) from a drone.

Before proposing the different approaches, it is also important to note that some aspects of this project were been fixed by the project’s corporate sponsor. The fixed aspects are as follows: the target spectrum for the project was 2.4 GHz, the software defined radio was the USRP B-200 Mini, and the antenna was the Alfa APA-M25.


\section{Drone Approaches}
For the approach, a multicopter was chosen to be the aerial platform as described in the previous chapter due to the increased maneuverability provided. The maneuverability will allow the platform to perform well in any target environment chosen for spectrum sensing. However, with the drone’s maneuverability, it sacrifices vital airtime and payload capacity, affecting design constraints that will be considered for each of the following approaches. The brainstorming process lead to four distinct designs that utilized a multicopter for spectrum sensing. These designs will be examined below.

\subsection{Approach One}
The first approach is centered around the 3DR Solo, a drone designed as a consumer drone used for aerial photography. The camera and gimbal were replaced with the modular spectrum sensing system to better utilize the limited payload capacity of this drone. The low payload capacity is a good representative of general drones for mounting the modular system, as most consumer based drones are not intended to carry heavy payloads.This drone has a payload capacity of 700 grams, which allows for a substantial payload. While the drone has a large payload capacity, the design will attempt to minimize the payload weight and enable the platform to be used on a larger number of drones. Furthermore, this drone can be configured to operate with the controller at 5.7 GHz that does not interfere with spectrum sensing at the 2.4 GHz wireless band. The transmissions back to the user would be on a frequency of 900 Mhz if real time response is required.
\subsection{Approach Two}
The second approach focuses on the DJI S1000 octocopter. The DJI S1000 was designed for professional photography and cinematography, where stabilization and high payload is essential as part of the design. Because of these qualities, this drone is priced higher than many consumer-level drones. This drone is able to carry payloads of up to 2 kilograms, providing large flexibility in terms of design constraints. However, because of the flexibility this high payload offers us, the resulting design may not be able to be used on consumer drones with a lower payload capacity. Another advantage this drone presents is that the communication between the controller and the drone is operated at 900 MHz, which will leave the detection of the 2.4 GHz wireless band unimpaired. As for implementation, this drone would serve better as a backup plan in case our modular system can not be lifted with the commercial drones used in testing.
\subsection{Approach Three}
The third approach focuses on using an autonomous navigation approach that was considered with the 3dr solo drone. One of the features that was provided with the drone is autonomous flight planning, where the user can set waypoints for the drone to fly. This feature will allow simplistic controls, however, general drones do not have this feature implemented. Furthermore, the 3dr solo drone’s ability to be customized is valued, as the initial controller-to-drone transmission frequency is in the target sensing frequency of 2.4 GHz. The autonomy of this design eliminates the need for constant communications with the drone while flying, preventing the control signals from interfering with the spectrum.
\subsection{Approach Four}
The fourth approach is focused on shielding the drone’s control antenna from the spectrum sensing antenna. This configuration was created to ensure that the wireless control signals received on the drone and the wireless spectrum being detected would not interfere with each other, since they are on the same frequency. Therefore, we would need to use a drone carrying 2 antenna configurations - one that would communicate with the ground, and the other that would perform spectrum sensing. The antennas would need to be isolated from each other in order to ensure this 2 antenna configuration works. This approach would help us have more flexibility in our drone platform and would let us have more flexibility on our communication protocols from the ground to the drone. The primary disadvantage to this approach is the additional weight that the shielding would add to the design. Due to our specific expertise and skill set of our group we concluded that this configuration for a drone would cause too much trouble to implement and also would not be reliable. 
\section{Onboard Computing}
Regardless of which drone approach was taken, an onboard processor and storage device had to be chosen for the processing and storage of IQ data. Single Board Computers (SBC) are a class of computer that are full computers on a single circuit board. The USRP B-200 Mini has a USB 3.0 interface, so a corresponding port on the SBC was required. The UP Board was the chosen processor because of the low power consumption. As for storage, a 32GB usb flash drive is capable of storing all the data gathered within 20 minutes.
%10/12/16 MHLI: May need more about other boards.
\section{Test Environments}
For this project, two main categories of test environments were selected for testing the aerial software defined radio: rural and urban areas . The best place to start initial testing would be in the rural areas, where there are limited interfering wireless signals. The testing will be done in a forested backyard so that the drone will be isolated from any irregularities which will provide a good resemblance of using the drone for search and rescue tasks. When testing in urban areas, the drone will be used mainly to collect IQ data to map hotspots, or eventually locate a specific wireless signal. However, testing in urban areas will require more planning due to air restrictions in cities as the drone can be a threat regarding privacy and possible physical collision. Urban areas are also more likely to have conflicting signals using the 2.4 GHz frequency band. The team will collaborate with Worcester Polytechnic Institute (WPI) or the Massachusetts College of Pharmaceutical and Health Sciences (MCPHS) to use their parking area in the city to test out the full system.
\section{Project Planning}
We used Microsoft Project to plan out our objectives throughout the project. The Gantt chart in Figure X shows our objectives and progress as of early October.
%10/12/16 MHLI: Need to include gantt chard and discussion of gantt chart.

