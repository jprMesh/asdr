\chapter{Proposed Approach}
This chapter will outline the approaches considered to accomplish this project. The goal of this project was to develop a modular platform for sensing the wireless spectrum through the use of a software defined radio (SDR) from an aerial platform. The aerial platform chosen is the multicopter drone due to its controllability and maneuverability. \par

The fixed aspects are as follows: the target spectrum for the project was 2.4 GHz, the software defined radio was the USRP B-200 Mini, and the antenna was the Alfa APA-M25. The software defined radio and antenna were donated to the project by Gryphon Sensors. The B-200 Mini has a small form factor or 83.3 x 50.8 x 8.4 mm, has a frequency range of 70MHz to 6GHz, is powered by USB (5V), and has an extensive set of libraries. It is an excellent choice for the spectrum sensing platform primarily due to its frequency range and small form factor. The antenna is designed to be used at 2.4 and and 5GHz, which are the two frequencies WiFi is transmitted at. The antenna is directional with a 16 degree vertical angle and a 66 degree horizontal angle. The directionality will allow for more accurate determination of the source of the WiFi signal.

\section{Prototype Systems}
A multicopter was chosen to be the aerial platform due to the increased maneuverability provided. Theismaneuverability will allow the platform to perform well in any target environment chosen for spectrum sensing. However, with the drone’s maneuverability, it sacrifices vital airtime and payload capacity, creating design constraints that will be considered for each of the following approaches. The brainstorming process led to four distinct design concepts that utilized a multicopter for spectrum sensing. \par

It should be noted that these four designs incorporate similar devices for the single board computer, storage, battery and GPS. Regardless of which drone approach was taken, an onboard processor and storage device had to be chosen for the processing and storage of IQ data. The single board was selected from five options which are shown in Appendix \ref{table:sbc_selection}. The final decision was to use the UP board. The decision was primarily between the UP board and the ODROID-XU4. Both of these boards have a similar form factor, but the UP board draws 1A less current. This reduces the power consumption of the board considerably and negates the lower weight of the XU4. An additional reason for choosing the UP board was the shipping location. The UP board was shipped from Europe, while the XU4 was shipped from Korea. Therefore, the lead time for the UP board would be shorter and documentation more accessible. As for storage, the selected device was a 32GB USB flash drive, which is capable of storing all the data gathered in one flight simulation of 20 minutes. \par

%Added a few more details about battery.
The external GPS and the battery were chosen for their small form factor. The GPS chosen was the GlobalSat ND-105C micro USB GPS receiver which has a form factor of 30.4mm x 15.4mm x 4.5mm. The battery chosen was the 11.1 Volt Lumenier 800mAh 3s 20c LiPo battery which had a form factor of 55mm x 31mm x 20mm. This battery, when stepped down to the voltage used by the UPboard, provides enough power for the computer to run 20 minutes at maximum power draw, the same amount of time as the drone can fly. \par

%Approach one
The first approach concept is based upon having an individual wireless communications frequency for each task. Initially this design was centered around the Autel X-Star that was controlled at 5.7 GHz band. However, upon learning that the 3DR Solo was configurable to operate at 5.7 GHz, and based on how well the 3DR Solo is documented, the approach was modified to center around the 3DR Solo. The 3DR Solo is a drone designed as a consumer drone used for aerial photography. The camera and gimbal were replaced with the modular spectrum sensing system to better utilize the limited payload capacity of this drone. This drone has a payload capacity of 700 grams, which allows for a substantial payload. While the drone has a large payload capacity, the design will attempt to minimize the payload weight and enable the platform to be used on a larger number of drones. Furthermore, this drone can be configured to operate with the controller at 5.7 GHz that does not interfere with spectrum sensing at the 2.4 GHz wireless band. The transmissions back to the user would be on a frequency of 900 Mhz if real time response is required \cite{3dr_Website}. More detailed design configurations on the first approach can be found in Appendix \ref{table:approach_one}. \par

%Approach 2
The second approach concept is focused on assuring that this project is feasible. This assurance is from the fact that the DJI S1000 octocopter was designed to carry heavier payloads. The DJI S1000 was designed for professional photography and cinematography, where stabilization and high payload is essential. As a result  of these qualities, as well as the more premium target market, this drone is priced higher than many consumer-level drones. This drone is able to carry payloads of up to 2 kilograms, providing large flexibility in terms of design constraints. However, because of the flexibility this high payload offers, the resulting design may not be able to be used on consumer drones with a lower payload capacity. Another advantage this drone presents is that the communication between the controller and the drone is operated at 900 MHz, which will leave the detection of the 2.4 GHz wireless band unimpaired. This approach serves better as a backup plan in case the modular system can not be lifted with the commercial drones used in testing. More detailed design configurations on the second approach can be found in Appendix \ref{table:approach_two}. \par
%Approach three
The third approach is based on using an autonomous navigation approach that was considered with the 3DR Solo drone. One of the features that was provided with the drone is autonomous flight planning, where the user can set way points for the drone to fly. This feature will allow simplistic controls, however, general drones do not have this feature implemented. Furthermore, the 3DR Solo drone’s ability to be customized is valued, as the initial controller-to-drone transmission frequency is in the target sensing frequency of 2.4 GHz. The autonomy of this design eliminates the need for constant communications with the drone while flying, preventing the control signals from interfering with the spectrum. More detailed design configurations on the third approach can be found in Appendix \ref{table:approach_three}. \par

%Approach four
The fourth approach is focused on shielding the drone’s control antenna from the spectrum sensing antenna. This configuration was created to ensure that the wireless control signals received on the drone and the wireless spectrum being detected would not interfere with each other, since they are on the same frequency. Therefore, a drone carrying 2 antenna configurations would be needed- one that would communicate with the ground, and the other that would perform spectrum sensing. The antennas would need to be isolated from each other in order to ensure the 2 antenna configuration works. This approach would help add resilience in the drone platform and would allow for flexibility on communication protocols from the ground to the drone. The primary disadvantage to this approach is the additional weight that the shielding would add to the design. More detailed design configurations on the fourth approach can be found in Appendix \ref{table:approach_four}. The table below compares and contrasts the four approaches.
\begin{table}[ht]
\centering
\caption{Pros and cons of the four approaches.}
\label{table:multi_pc}
\begin{tabular}{|l|p{6cm}|p{6cm}|}
	\hline 
  Approach & Pros & Cons \\ \hline
  1 & No interference & Minimal payload capacity \\
  2 & Great payload capacity \& No interference & Not a commercial drone \\
  3 & Does not require communication signals & Flight path has to be predetermined \\ 
  4 & N/A & Increased payload for sheilding \\ \hline
\end{tabular}
\end{table}

\section{Project Planning}
Microsoft Project was used to plan out objectives throughout the project. The Gantt chart in Figure \ref{fig:gantt_chart} shows the project objectives. The planning came in three parts: implementation, testing and writing the report. In the initial stages of implementation, each individual part was done first. When each part worked separately, the blocks were then integrated to make sure that the system functions as planned. From then on, the project development moved to the testing stage, where the system was tested in three scenarios. These scenarios allowed ease of debugging if the testing does not go according to plan. The SDR was tested to determine whether it can locate a WiFi signal without any movement. Then, before mounting the system on the drone, the system was moved around to check whether the data correlates to a specific direction of a transmitter when the system is mobile. Finally, when both tests worked properly, the system was mounted on the drone and tested while flying. During the process of implementation and testing, the report was be written based on the work done in the implementation and testing stages.
\begin{figure}[ht]
\centering
\includegraphics[width=\textwidth]{img/oct_gantt_chart.png}
\caption{Project planning gantt chart. This chart was used to display deadlines and ensure that tasks were being completed. Team members were assigned tasks and were held accountable for those tasks.}
\label{fig:gantt_chart}
\end{figure}
