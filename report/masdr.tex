% FIRST OF ALL:
% If you are using X-based Emacs to read this file, please switch on
% Syntax Highlighting by typing:
%    ALT-X  font-lock-mode   (or META-X on X-Terminals)
%
% That should make these comments nice and red so they can be easily
% distinguished from the actual code. 
% =======================================================================

% This is a template for  Masters' Theses at WPI.
% It complies (more or less) to the standards given by the Library 
% (as of February 1999)
%
% Feel free to use this file, but I give no guarantee for its compliance
% to standards (meaning I won't pay for the paper if the library rejects it :))
%
%
% The lengths (textheight, width etc.) are fine-tuned for ps1, ps2, and ps3, 
% but seem to be somewhat dependent on the machine you are using to compile, 
% the date, time, moon phase, the weather, and other quantum effects.
% You may have to change \oddsidemargin a little, but it's about 98% correct.
%
% Also, the spacing is correct (doublespacing with footnotes correctly
% singlespaced). Curiously, the font size is not specified in the
% regulations. So feel free to change it, but the majority of theses
% that I have seen is written in 12 point font.
% 
% As for the inclusion of graphics, I recommend the methods specified
% in ``latexguide.ps'' off the CS-GSO Website. You can use other
% methods including copy and paste with a photocopier, but I think
% using the graphicx package is the easiest.
%
% Have fun and good luck with the thesis.
%
%
%  Andreas Koeller (koeller@wpi.edu)
%
%
%

% The preamble
%
%
% 12 point font, and your thesis is a ``report'' to LaTeX
\documentclass[12pt]{report}

% this enables correct linespacing and graphics inclusion via 
%``\includegraphics''
\usepackage{setspace}
\usepackage{graphicx}


% leave 1.5in margin to the left and 1in margin to the other
% sides. Don't print page number in the margin (but rather above it)
\setlength{\textheight}{8.63in}
\setlength{\textwidth}{5.9in}
\setlength{\topmargin}{-0.2in}
\setlength{\oddsidemargin}{0.3in}
\setlength{\evensidemargin}{0.3in}
\setlength{\headsep}{0.0in}

% Start to write
\begin{document}
	
	% First things first: The Titlepage
	% This is the recommended format by the library
	%
	
	
	% Define \brk as a command for leaving a little vertical space. Makes
	% the titlepage easier to read - normally, this is NOT GOOD LATEX
	% STYLE!!!
	%
	\newcommand{\brk}{\vspace*{0.18in}}
	
	% No page number on the title page
	\thispagestyle{empty}
	
	% Center the whole title page
	\begin{center}
		
		\brk
		
		% Large font and bold face for the headline. Try to keep it at one or
		% two lines. Headlines over two lines will mess up the spacing, and you have to
		% manually finetune it. Note that the line break in the SOURCE CODE
		% does not affect the line breaking in the output file. If you want
		% hardcoded line breaks, you have to mark them with a double backslash (\\)
		
		{\large 
			\textbf{
				Aerial Software Defined Radio
			}
		}
		
		
		\brk
		by
		
		\brk
		% insert your name here. 
		Jane Doe
		
		
		% All this is constant:
		\brk\brk
		A Thesis
		
		\brk
		Submitted to the Faculty
		
		\brk
		of the 
		
		\brk
		WORCESTER POLYTECHNIC INSTITUTE
		
		\brk
		In partial fulfillment of the requirements for the
		
		\brk
		Degree of Master of Science
		
		\brk
		in
		
		\brk
		Electrical and Computer Engineering
		
		\brk
		by
		
		% This is how LaTeX draws lines :) It's where your signature goes.
		\brk\brk
		\rule{3in}{1.2pt}
		
		% Adjust this to your preferred month and year
		\brk
		May 2000
		
	\end{center}
	
	
	\vfill
	APPROVED:
	
	\vspace{0.5in}
	\rule{3in}{0.8pt}
	
	% Change this to your favorite CS professor.
	Professor John Doe, Major Thesis Advisor
	
	\vspace{0.5in}
	\rule{3in}{0.8pt}
	
	% This is also constant :)
	Professor Micha Hofri, Head of Department	
	
	
	% end of titlepage
	\newpage
	
	% This is the command for doublespacing when you use the setspace
	% package
	% Please do NOT use \baselinestretch, this will mess up everything,
	% cause earthquakes, tornados and lots of questions for me...
	% If you need a singlespaced paragraph (BAD STYLE!!!), use
	% \singlespacing or \onehalfspacing and enclose it together with the
	% paragraph in braces {\singlespacing This is my text... blah blah blah}
	%
	\doublespacing
	
	% Now you can start to be creative.
	% First, you need an abstract.
	% Fortunately, LaTeX has thought of that, so it's very easy:
	%
	\begin{abstract}
		This paper is the most important paper I have ever written. Therefore,
		everyone should read it, like it, and recommend it to all his friends.
	\end{abstract}
	
	% From here on, we need Roman page numbers according to the library
	% regulations. So let's assign those.
	
	\pagenumbering{roman} % or {Roman} if you like them capitalized
	
	% The next thing is the Preface (``Acknowledgements'').
	% No standard environment for that, so we'll format it by hand.
	%
	\begin{center}
		\textbf{Acknowledgements}
	\end{center}
	
	I would like to express my gratitude to my advisor who made
	sure the thesis has at least 120 pages, 200 pictures and lots
	of formulae and thus made me master \LaTeX{} like my native
	language.
	
	My thanks are also due to my reader... who has read the thesis
	in the two days that I gave him since it wasn't done until two
	days before due date.
	
	Thanks also to ... lots of friends, the fact that a week has
	seven days instead of only five as I had always thought, and
	the fact that I own a key to the building so I can work at four
	in the morning whenever I feel like it. That is, all the time.
	
	% P.S. You don't have to add me to the acknowledgements for providing 
	% this file :)
	
	\clearpage
	
	
	% Now comes the Table of Contents, really easy in LaTeX. you never
	% have to worried about it. (Think of all the hours you would
	% have wasted in Word getting this thing updated without crashing
	% the system) :).
	
	\tableofcontents
	
	% THAT'S IT. REALLY. Everything else is automatic. No formatting, no headline.
	% All predefined.
	
	% Now - just as easy - the List of Figures.
	% This will catch all objects enclosed in \begin{figure}\end{figure}
	% statements.
	\listoffigures
	
	% There is also a list of tables, if you have any.
	% This will catch all objects enclosed in \begin{table}\end{table}
	% statements.
	\listoftables
	
	
	% And we need a clear separation between preface and text, otherwise
	% the numbering gets confused.
	
	\clearpage
	
	% And now - tataa - the text.
	% This is the place to become really creative.
	
	% From here on, we need arabic numbering again and we need to start
	% from 1.
	
	\pagenumbering{arabic}
	\setcounter{page}{1}
	
	% 
	% Since this is a ``report'', the topmost level of hierarchy is
	% ``Chapter'', not section as you may be used to. Chapters are
	% enumerated starting from 1, so Sections are 1.1, Subsections are
	% 1.1.1, subsubsections don't get numbers. (You can change that, if
	% you want them to be called 1.1.1.1)
	%
	\chapter{First Chapter.}
	This should ideally contain some text.
	\section{First Section.}
	More Text.
	\subsection[Alternative title for the Table of Contents]{First Subsection.}
	Even more text, maybe a formula:
	\begin{equation}
	\sum_{i=1}^{n}i=\frac{n(n+1)}{2} % much easier than Microsoft Equation
	% Editor :)
	\end{equation}
	\subsubsection{First Sub-subsection.}
	This is really deep down in the hierarchy. Maybe you shouldn't even
	use sub\-subsections. It goes further down (paragraphs), but I don't
	think you'll need that\footnote{By the way: notice that, although we
		have doublespacing here, the footnotes are singlespaced. This is
		intended and good. If you want to change that, try, but this is really
		how it should be.}.
	
	\section{Other thoughts.}
	Okay, what else?
	Let me quickly put a figure here, maybe a piece of pseudo code.
	That way, you can see how this is done. It's a little painful, but
	looks really cool. We will call it Figure~\ref{fig:source_algo1}. The
	numbering is automatic---don't worry about it.
	
	%First, we want to make our life easier and define a ``TAB'' command.
	\newcommand{\T}{\hspace*{5mm}}
	
	% Start a figure
	\begin{figure}[htb]
		% We would like to have the whole thing in the center of the page
		\begin{center}
			% We want a frame. 
			\fbox{
				% The figure should autoformat to half the page width
				\begin{minipage}{0.5\textwidth}
					% Now comes the content.
					% For source code, you have to leave an empty line after each line of
					% code.
					% Note that this is text-mode, that's why all formluae are typeset in
					% math-mode (enclosed in dollar-signs $a+b$)
					% Each line needs a font command a la \texttt{}, \textsc{}, textbf{}
					% You could use \begin{verbatim}\end{verbatim} for source code, but then 
					% you can't do any more formatting in you source file. May be appropriate
					% sometimes. 
					
					\textsc{Bellman-Ford} $(G,w,s)$
					
					(1) \textsc{Initialize-Single-Source}$(G,s)$
					
					(2) \textbf{for} $i\leftarrow 1$ \textbf{to} $|V[G]|-1$ \textbf{do}
					
					(3) \T\textbf{for} each edge $(u,v)\in E[G]$
					
					\T\T\T\textbf{do} \textsc{Relax} $(u,v,w)$
					
					(4) \textbf{for} each edge $(u,v) \in E[G]$
					
					\T\T \textbf{do if} $d[v]>d[u]+w(u,v)$
					
					\T\T\T \textbf{then return} \textsc{false}
					
					(5) \textbf{return} \textsc{true}
					
					% End of content, closure of minipage, frame, and centering.
				\end{minipage}
			}
		\end{center}
		% Caption
		\caption{This is a very simple algorithm in pseudocode.}
		% A label to refer to the figure.
		\label{fig:source_algo1}
		% End of figure
	\end{figure}
	
	\noindent
	And so on, and so on.
	
	Please remember that you have to compile a document several times when
	you did changes that affect figures, table of contents, bibliography,
	etc. (This is always the case if you get the warning: ``LaTeX Warning:
	Label(s) may have changed. Rerun to get cross-references right.'').
	
	The recommended sequence is :
	
	\texttt{latex foo.tex}
	
	\texttt{bibtex foo}
	
	\texttt{latex foo.tex}
	
	\texttt{latex foo.tex}
	
	% Let's assume this is the end of your thesis text.
	
	% Now come appendices, if you had any.
	% Appendices are automatically numbered, just like everything else in
	% LaTeX. But only after you gave this command
	\appendix
	
	\chapter{More to say}
	
	\section{A section within an appendix.}
	This is an appendix.
	
	
	% Last and least (at least, that's what the library says) - the
	% Bibliography.
	
	
	% you can save some space by having the bibliography singlespaced, if you want
	\singlespacing
	
	%
	% You should become familiar with the BibTeX program, which
	% uses a *.bib-file to collect all citations that you have. It's a lot
	% prettier than typing all the citations right into the document. The 
	% reference to citations also works well that way, but the exact 
	% explanation of that will be on the CS-GSO homepage, whenever I'll ever 
	% have time for that.
	%
	%
	% If you use BibTeX, the bibliography is very easy. You refer to
	% citations in the text with \cite{tag}, where tag is the tag that you
	% defined in the bib-file.
	% Then, you run bibtex once in a while during compilation, and the
	% rest is done in two lines:
	
	
	\bibliographystyle{alpha}
	\bibliography{foo}
	
	% which assumes a file foo.bib in your working directory.
	% The word ``Bibliography'' will appear in your document as soon as
	% you used ``bibtex'' on the command line.
	%
	% For reference on this, refer to the CS-GSO homepage.
	
	%============================
	%That's all, folks. Have fun.
	%
	%                     Andreas
	%============================
	
	
\end{document}









