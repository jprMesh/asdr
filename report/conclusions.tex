\chapter{Conclusions}
The technologies of Cognitive Radios and Quadcopters are encountering a renaissance
with many different projects exploring the possibilities that they introduce. In this project,
the two technologies were combined, resulting in a versatile platform. By doing so, 
the team achieved the following:
%Don't forget to add more details.
\begin{itemize} 
 \item \textbf{Constructed hardware platform: }Constructed a self-contained SDR signal processing
                unit, able to operate separate from the drone used.
 \item \textbf{Created software platform: }Created the MASDR software design platform,
                written to be flexible and easy to modify. 
 \item \textbf{Collected data: }Gathered assorted information from the air, including 
                IQ samples, matched filter-processed samples, RSS values, and GPS locations.
 \item \textbf{Communication to ground tested: }A DBPSK transmitter and receiver, 
                including Root Raised Cosine Filter and channel equalization were created and tested. 
 \item \textbf{Kalman Filter modeled in MATLAB: }A four-state tracking Kalman Filter
                was modeled in MATLAB, using a constant-velocity model.
\end{itemize} \par
While designing MASDR, a number of problems were encountered. The team came to the following 
conclusions:
\begin{itemize}
 \item \textbf{Transmission and Receiving should both be done with either UHD or GNURadio.} 
                GNURadio provides a lot of useful blocks that make designing a receiver easier than in
                UHD, where every block has to be written. However, this means that a lot of the behind-the-scenes
                work in GNURadio isn't obvious, and has to be deeply investigated. Either use the premade blocks 
                in GNURadio or design the whole system with UHD.
 \item \textbf{GPS is only good for coarse localization.} 
                GPS has a low refresh rate in comparison to the 
                sample rate of the USRP B200 mini. In addition, it only has an precision of
                a few meters. In order to get a more precise location and estimate,
                other sensors, like inertial measurement units (IMU), need to be used. 
                This introduces a whole new layer of data fusion.
 \item \textbf{A smarter WiFi Channel decision is needed.} 
                In the MASDR implementation, the SDR
                only looks at one WiFi channel, in order to allow for accurate RSS measurements. 
                This channel is hard coded into the system. In order to have a more encompassing 
                WiFi sensing and localizing solution, a smarter way to sense WiFi channels is 
                necessary.
\end{itemize} \par

\section{Future Work}
The primary focus of this MQP was to create an aerial SDR testbed, on which more 
specialized projects can be completed. This is the primary area where future work
could be based. One possibility is to investigate the localization of the controller
of a drone. This involves locating a communication signal from a drone, determining its
SSID, finding another transmitter with the same SSID, and localizing it. Another possibility
is to use two aerial SDR platforms to empirically model a channel. One platform
would act as the transmitter, the other as the receiver. These are two of many possibilities.
With the growth of Cognitive Radio and quadcopters, the number of possible applications
will only increase.\par

One of the secondary purposes of this MQP was to implement WiFi localization on 
the platform. There are many different ways to do this, covered in the background.
This project focused primarily on using a matched filter in the frequency domain to match
an OFDM header used by most WiFi communications, but WiFi is only one of many 
signals to be identified. In looking at other signals, other energy detection
techniques may be warranted. In addition, the platform can be modified to 
classify the signals it receives.\par

The Kalman filter used in this project was designed for GPS values, and had a few shortcomings.
The first shortcoming was that it was only implemented in MATLAB, using a very basic
test signal and simplistic model. There is C++ code that was written, but never tested.
The first step would be to use the Kalman filter in the existing code. Beyond this,
the model used to generate the Kalman gains can be improved, which would result in better 
predictions. In addition, a Kalman filter could be applied to other aspects of the 
system, such as the RSS sensing value, and the distance calculation that uses it.
This is non-trivial, because the equation for getting distance from RSS is non-linear,
so an Extended Kalman filter would be necessary. 
