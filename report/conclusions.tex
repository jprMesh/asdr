\chapter{Conclusions}
In this chapter, the accomplishments acheieved by this MQP, as well as ways that 
the project can be expanded in the future. 
 * Made the system
 * flew the thing
 * logged I, Q data
 * Match filtered.
 * RSS value.
 * GPS value
 * Tx/rx working on ground.
 * Kalman Filter in MATLAB

\section{Conclusions} %Change when we have a discussion on it.
 * Tx/Rx is more complicated than expected.
 * GPS is only good for coarse localization.

\section{Recommendations}
 * Smarter channel measuring/deciding (instead of just reading one).
 * TX/RX in air
 * Kalman Filter Test/implement in C.
 * 5 GHz drone ctl


\section{Future Work}
The primary focus of this MQP was to create an aerial SDR testbed, on which more 
specialized projects can be completed. This is the primary area where future work
could be based. One possibility is to investigate the localization of the controller
of a drone. This involves locating a communication signal from a drone, determining its
SSID, finding another transmitter with the same SSID, and localizing it. Another possibility
is to use two aerial SDR platforms to emperically model a channel. One platform
would act as the transmitter, the other as the receiver. These are two of many possibilities.
With the growth of Cognative Radio and quadcopters, the number of possible applications
will only increase.\par

One of the secondary purposes of this MQP was to implement WiFi
In addition to making the platform, the MQP focused on energy detection 
and localization. 
 * Other energy detection techniques
 * Classification. \par

A Kalman filter was implemented in matlab for motion tracking. It was kind of 
working, worked in matlab. 
 * (finish) port to c++
 * test experimentally
 * EKF
 * KF other aspects.
 * Incoroporate other sensors.