\chapter{Conclusions}
In this chapter, the accomplishments acheieved by this MQP, as well as ways that 
the project can be expanded in the future. 
 * Made the system
 * flew the thing
 * logged I, Q data
 * Match filtered.
 * RSS value.
 * GPS value
 * Tx/rx working on ground.
 * Kalman Filter in MATLAB

\section{Conclusions} %Change when we have a discussion on it.
 * Tx/Rx is more complicated than expected.
 * GPS is only good for coarse localization.

\section{Recommendations}
 * Smarter channel measuring/deciding (instead of just reading one).
 * TX/RX in air
 * Kalman Filter Test/implement in C.
 * 5 GHz drone ctl


\section{Future Work}
The primary focus of this MQP was to create an aerial SDR testbed, on which more 
specialized projects can be completed. This is the primary area where future work
could be based. One possibility is to investigate the localization of the controller
of a drone. This involves locating a communication signal from a drone, determining its
SSID, finding another transmitter with the same SSID, and localizing it. Another possibility
is to use two aerial SDR platforms to emperically model a channel. One platform
would act as the transmitter, the other as the receiver. These are two of many possibilities.
With the growth of Cognative Radio and quadcopters, the number of possible applications
will only increase.\par

One of the secondary purposes of this MQP was to implement WiFi localization on 
the platform. There are many different ways to do this, covered in the background.
This project focused primarily on using a matched filter in the frequency domain to match
an OFDM header used by most WiFi communications, but WiFi is only one of many 
signals to be identified. In looking at other signals, other energy detection
techniques may be warranted. In addition, the platform can be modified to 
classify the signals it receives.\par

The Kalman filter used in this project was designed for GPS values, and had a few shortcomings.
The first shortcoming was that it was only implemented in MATLAB, using a very basic
test signal and simplistic model. There is C++ code that was written, but never tested.
The first step would be to use the Kalman filter in the existing code. Beyond this,
the model used to generate the Kalman gains can be improved, which would result in better 
predictions. In addition, a Kalman filter could be applied to other aspects of the 
system, such as the RSS sensing value, and the distance calculation that uses it.
This is non-trivial, because the equation for getting distance from RSS is non-linear,
so an Extended Kalman filter would be necessary. 
