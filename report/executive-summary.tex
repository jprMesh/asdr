\chapter*{Executive Summary}
Advancements in technology have paved a path towards a safer and more convenient lifestyle. This project focuses upon combining two recent technologies, the drone and software defined radio, to ultimately create a modular platform for wireless signal mapping, search and rescue purposes, and localization of controllers of unregistered drones. \par

To achieve the desired goal, the software defined radio had to be mounted securely on the drone, flown around, while sensing the wireless signals around the area. This idea leads to the scope of this project, which consists of background research into prior art, the implementation of the platform engineering design, and testing as a proof of concept. \par

Through the research of prior art, it was determined that similar projects have been conducted on these individual technologies, however not as much with them combined. Examples of prior art are the Sensor Processing and Path Planning Framework for a Search and Rescue UAV Network \cite{path_planning_snr_mqp} and the Software Defined Radio Localization using 802.11-style Communications \cite{sdr_localization_mqp}. This provided that the research done in the project will be the foundation of future development. \par

The implementation of this system can be seperated into multiple blocks. These blocks consist of coding framework, drone control, spectrum sensing, spectrum localization and transmission to ground. Due to the independence of the modular platform to the drone, the coding framework is used to describe the actions of the system dependent on the state of the drone. These actions consist of sensing, rotation, on board computing and transmission to ground. In terms of drone control, most commercial drones operate at 2.4GHz which is the targetted sensing spectrum for this project. This is an issue because of the possibility that the drone control signals can interfere with the signals that the antennas are receiving. Therefore the wireless cards have to be replaced so that these drones can operate at 5GHz, a different frequency to the sensing spectrum. To conduct spectrum sensing, the board had to be interfaced with the Universal Software Radio Peripheral hardware driver (UHD) to the Ettus board. This required knowledge of how these libraries worked and how information is parsed in order to perform matched filtering. For spectrum localization, the board had to also be interfaced with a Global Positioning System (GPS) that determines the device's location. To improve accuracy, the GPS was implemented with a Kalman Filter as localization is highly dependent on accurate position information.










The documentation provided can be used as guidelines for future development.